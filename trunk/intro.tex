\begin{abstract}

O presente trabalho visa demonstrar as camadas do modelo OSI (Open Systems Interconnection) e o funcionamento e as vantagens de se utilizar um sistema de controle de versões no desenvolvimento de projetos.

\end{abstract}

\newpage

\section{Introdução}

O OSI  ou Interconexão de Sistemas Abertos é um modelo usado para entender como os protocolos de rede funcionam. Para facilitar a interconexão de sistemas de computadores, a ISO (International Standards Organization) desenvolveu um modelo de referência chamado OSI (Open Systems Interconnection) para que os fabricantes pudessem criar protocolos a partir deste modelo.

O modelo OSI é dividido em sete camadas. Cada camada é responsável por algum tipo de processamento e cada camada apenas se comunica com a camada imediatamente inferior ou superior.

Subversion é um sistema de controle de versões de código aberto. Um sistema de controle de versões é um software utilizado para auxiliar no desenvolvimento de projetos. Gerenciando arquivos e diretórios, é possível obter diversas informações sobre o andamento do desenvolvimento do projeto.s