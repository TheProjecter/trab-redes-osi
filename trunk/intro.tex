\begin{abstract}

O presente trabalho visa demonstrar a aplicação das camadas OSI no protocolo IRC (Internet Relay Chat).

% força aí Alessandro!!!

\end{abstract}

\newpage

\section{Introdução}

O OSI  ou Interconexão de Sistemas Abertos é um modelo usado para entender como os protocolos de rede funcionam. Para facilitar a interconexão de sistemas de computadores, a ISO (International Standards Organization) desenvolveu um modelo de referência chamado OSI (Open Systems Interconnection) para que os fabricantes pudessem criar protocolos a partir deste modelo. \cite{TORRESOSI}

O modelo OSI é dividido em sete camadas. Cada camada é responsável por algum tipo de processamento e cada camada apenas se comunica com a camada imediatamente inferior ou superior. \cite{VENTURAOSI}

O servidor de mensagens IRC é analisado e classificado em cada uma das sete camadas do modelo OSI.

O IRC é um protocolo que permite a comunicação  entre usuários. “O IRC fornece um modo de comunicação em tempo real com pessoas de todo o mundo. Consiste em várias redes separadas ("networks" ou "nets") , de Servidores de IRC, os quais permitem aos utilizadores ("users") ligarem-se ao IRC.”\cite{INTROIRC}