\subsection{Cenário anterior}

Como eram desenvolvidos os sistemas antes da implantação do controle de versões?

Há uma equipe de 4 programadores, sendo 1 responsável pela publicação das mudanças nos projetos principais. Cada desenvolvedor copiava o projeto inteiro que estava no servidor para a sua máquina. Esse processo é repetido todas as manhãs.

Tendo uma cópia de trabalho em sua máquina, o desenvolvedor fazia as tarefas necessárias (manutenção de módulos, criação de módulos, etc) e enviava as mudanças (por e-mail ou via rede) para o desenvolvedor responsável publicá-las no sistema.

Nesse processo, há os seguintes problemas:

\begin{itemize}
\item Não há um histórico informando, por exemplo, quem alterou o seguinte módulo, quando alterou, e porque alterou, bem como informações no caso da criação de novos módulos e funcionalidades. Em caso de problemas decorrentes dessas alterações, sem um histórico desse tipo o tempo demandado para identificar o problema pode ser maior do que o tempo gasto no caso de se ter um histórico detalhado das mudanças.

\item O risco de erro na cópia dos projetos não deve ser descartado. A cópia errada de arquivos pode sobrescrever um módulo ainda não finalizado, pondo a perder as mudanças a serem implantadas no sistema.
\end{itemize}
