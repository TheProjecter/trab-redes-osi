\documentclass{beamer}
\usepackage[utf8]{inputenc} 	% reconhece a acentuação para o português
\usepackage[brazil, english, portuguese, brazilian]{babel}
\usepackage{hyperref}		% para que o arquivo eletronico tenha links
\usepackage{aecompl} 		% aumentar qualidade das fontes
\usepackage{amsmath, amsfonts, epsfig, xspace}
\usepackage{pstricks,pst-node}
\usepackage[normal,tight,center]{subfigure}
\setlength{\subfigcapskip}{-.5em}
\usepackage{beamerthemesplit}
\usepackage{graphicx} 		% para poder utilizar imagens
\usepackage{float} %Inserir fig flutuantes use opcao H ficara na posicao que voce inseriu
\usepackage{wrapfig} %para colocar figuras do lado de Texto
\usepackage{verbatim} %para colocar codigo LATEX sem que seja executado
\usepackage{pgf} %Logo nos Slides

\usetheme{Madrid} %Algus temas: Warsaw, Berkeley, Madrid
%\usetheme{lankton-keynote} %Tema 2

\date{novembro de 2008}
\author[Maycon Santos\\Paulo Deininger\\Prof.Dr.Palin]{Maycon dos Santos Nunes\\Paulo Eduardo Deninger Messias Alves\\Prof. Dr. Marcelo Facio Palin}

\institute
{
  \texttt{<maycon261@gmail.com\\paulodeininger@gmail.com\\profpalin@gmail.com>}\\
  .\\%Pula Linha
  \url{http://www.drumond.com.br/}\\
  Faculdade Drummond.
}
\subject{Computação Paralela}

\pgfdeclareimage[height=8mm]{logo}{img/logo}
\pgfdeclareimage[height=7mm]{cc}{img/cc}
\pgfdeclareimage[height=7.17cm]{cc-latex}{img/cc-latex}

\logo{\pgfuseimage{logo}}

\begin{document}

%%%%%%%%%%%%%%%%%%%%%%%%%%%%%%% SLIDE - Título e Autor %%%%%%%%%%%%%%%%%%%%%%%%%%%%
%Titulo abreviado seguido do Titulo Completo
\title[Trabalho de Conclusão]{Computação Paralela}
\frame{
\titlepage
\pgfuseimage{cc}
}

%%%%%%%%%%%%%%%%%%%%%%%%%%%%%%% SLIDE - Sumário %%%%%%%%%%%%%%%%%%%%%%%%%%%%
\begin{frame}{Sumário}
\tableofcontents
\end{frame}

%%%%%%%%%%%%%%%%%%%%%%%%%%%%%%% SLIDE %%%%%%%%%%%%%%%%%%%%%%%%%%%%
\section{Introdução}

\begin{frame}{Introdução}
  Computação Paralela
  \begin{itemize}
    \item O que é computação paralela?
    \item Para que serve?
    \item Computador paralelo
    \item Utilização na atualidade
  \end{itemize}
\end{frame}

%%%%%%%%%%%%%%%%%%%%%%%%%%%%%%% SLIDE %%%%%%%%%%%%%%%%%%%%%%%%%%%%
\section{Programação Paralela}

\begin{frame}{Programa\c{c}\~ao Paralela}
  \begin{itemize}
    \item O que é Programação Paralela?
    \item Quando é viável a sua utilização?
    \item Linguagens suportadas
       \item Bibliotecas:
          \item MPI-2
	  \item OpenMP
  \end{itemize}
\end{frame}

%%%%%%%%%%%%%%%%%%%%%%%%%%%%%%% SLIDE %%%%%%%%%%%%%%%%%%%%%%%%%%%%
\section{Diferenças MPI-2 x OpenMP}
\begin{frame}{Diferenças MPI-2 x OpenMP}
\textbf{Programação Multithread}
  \begin{itemize}
       \item MPI-2 
       \item OpenMP 
  \end{itemize}
\end{frame}

%%%%%%%%%%%%%%%%%%%%%%%%%%%%%%% SLIDE %%%%%%%%%%%%%%%%%%%%%%%%%%%%
\section{Conclusão} % add these to see outline in slides
\begin{frame}
  \frametitle{Bibliografia}
  \begin{itemize}
\item Especificação MPI 1. \\
  \url{<http://www.mpi-forum.org/docs/mpi-11-html/mpi-report.html>}
\item Especificação MPI 2. \\
 \url{<http://www.mpi-forum.org/docs/mpi-20-html/mpi2-report.html>}
\item PALIN, M. F. Técnicas de decomposição de domínio em computação paralela para a simulação de campos eletromagnéticos pelo método dos elementos finitos. 2007.
 \item PALIN, P. D. M. F. Compilando e testando a biblioteca mpi. 2008.\\
Disponível em: \url{<http://www.insciti.com.br/ead/moodle/mod/resource/view.php?id=63>}.
  \end{itemize}
\end{frame}

\begin{frame}
  \frametitle{Bibliografia cont.}
  \begin{itemize}
\item AZEREDO, P. Aulas ministradas na ufsc - curso de c. 2008. \\
 Disponível em: \url{<http://mtm.ufsc.br/~azeredo/cursoC/aulas/c810.html>}
\item CHARAO, P. A. S. Análise de desempenho de programas paralelos. 2006.
\item CORPORATION, M. Openmp no visual c++. 2008.\\
 Disponível em: \url{<http://msdn-ıvel.microsoft.com/pt-br/library/tt15eb9t(VS.80).aspx>}
\item MAIA, L. P. Multithread. 1998.
  \end{itemize}
\end{frame}

%%%%%%%%%%%%%%%%%%%%%%%%%%%%%%% SLIDE %%%%%%%%%%%%%%%%%%%%%%%%%%%%
\begin{frame}
  \frametitle{Conclusão}
  \begin{center}
   \large \textbf{Conclusão}
  \end{center}

\end{frame}

%%%%%%%%%%%%%%%%%%%%%%%%%%%%%%% SLIDE %%%%%%%%%%%%%%%%%%%%%%%%%%%%
\begin{frame}
  \frametitle{Questões}
  \begin{center}
   \large \textbf{Obrigado!}
  \end{center}

\end{frame}

%%%%%%%%%%%%%%%%%%%%%%%%%%%%%%% SLIDE %%%%%%%%%%%%%%%%%%%%%%%%%%%%
\begin{frame}

  \frametitle{Creative Commons}
  \begin{center}
    \pgfuseimage{cc-latex}
  \end{center}
\end{frame}

\end{document}